\chapter{Conclusion and future work} % Conclusion
\label{ch:sum}


This thesis presented a solution to automate the detection of three different types of bugs in C and C++ codebases.
We developed two checkers for Clang Static Analyzer, an open-source static analysis tool.
The project was a success, and it met all of the original specifications.
One of the checkers has already passed the LLVM code review, demonstrating its high quality, and runs on millions of devices across the world as part of the Clang. Thesis documentation can be used as a guide by anyone who wishes to develop Clang Static Analyzer checkers. 

% In this thesis, we presented a solution to automate the detection of three different types of bugs in C and C++ codebases. We have created two checkers for the open-source static analysis tool - Clang Static Analyzer. The project was successful and meets the demands of all of the original specifications. One of the checkers already passed LLVM's code review, proving it's highest quality, and runs on millions of devices across the world as part of the Clang. 

Both checkers can be improved.
The InvalidPtr checker's short-term goal would be to successfully complete the LLVM review, later investigation should follow to extend it with more non-reentrant functions that suffer from similar problem, whereas the POS34 checker's goal would be to remove it from the so-called "alpha" state, i.e. enable it by default for all Clang Static Analyzer runs.
This could be accomplished by reducing the number of false positives, for example, by warning on the last \lstinline{putenv()} call on the execution path through the current stack frame.


% Both checkers can be further improved. Short term goal of the InvalidPtr checker would be to successfully conclude the review, while for POS34 checker aim is to take it out from the so-called "alpha" state, i.e. enable it by default for all Clang Static Analyzer runs. This could be achieved by reducing the number of false positives, for example, by warning on the last \lstinline{putenv} call on the execution path through the current stack frame.

Overall, this project is not only my favorite topic in program analysis, but it is also an excellent choice for demonstrating knowledge gained throughout my bachelor's degree studies.


\section*{Acknowledgements}
I would like to thank the following people, without whom I would not have been able to complete this thesis, and without whom I would not have made it through my bachelor degree! 

First and foremost, I'd like to thank my supervisor, Prof. Zoltán Porkoláb, for assisting me in selecting a thesis topic and guiding me through the process of writing thesis work. 

I would also like to extend my deepest gratitude to to Balázs Benics (ELTE, Ericsson) and Csaba Dabis (ELTE) for assisting with the design of the checkers, as well as answering all of my questions during the development phase, dedicating time for one-on-one meetings even on weekends, debugging the code with me, and suggesting how to further polish the work for the LLVM submission. 

Special thanks to Ericsson's CodeChecker team for their valuable suggestions during the early phase of the checkers. And to my team at Ericsson -- CI Joe, for giving me the flexibility and time needed to work on this thesis. 

Many thanks to Prof. Zoltán Gera for mentoring and introducing me to the Clang in Security Checker Development lab, as well as for teaching Imperative Programming during the first semester of my bachelor studies, the subject that served as the foundation for my C++ and C knowledge. 

I would also like to acknowledge Prof. Viktoria Zsók, who first introduced me to doing research early at my bachelor studies. Writing a thesis was made much easier because of previous experience. 

I am also grateful to LLVM reviewers: Artem Dergachev (Apple Inc.), Kristóf Umann (ELTE, Ericsson), and Aaron Ballman (Intel Corp.), for their insightful comments on the code and for making it possible that part of my thesis runs on millions of devices worldwide as part of Clang.
